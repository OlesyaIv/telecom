\documentclass[10pt,a4paper,oneside]{article}
\usepackage[utf8]{inputenc}
\usepackage{amsmath}
\usepackage[english, russian]{babel}
\usepackage{graphicx}
\graphicspath{{pictures/}}
\usepackage{multicol}
\RequirePackage{float}
\usepackage[left=2cm,right=2cm,top=2cm,bottom=2cm]{geometry} 
\usepackage{listings}
\setcounter{totalnumber}{10}
\setcounter{topnumber}{10}
\renewcommand{\topfraction}{1}
\renewcommand{\textfraction}{0}
\lstdefinestyle{Style}{
	numbersep=10pt,
	frame=single,
	captionpos=b
}
\lstset{style=Style}
\begin{document} 
\begin{titlepage}
		\newpage
		\begin{center} % Размещение ткста - по центру
			Санкт-Петербургский политехнический\\ 
			университет Петра Великого\\
			Институт компьютерных наук и технологий\\
			Кафедра компьютерных систем и программных технологий\\
			\vspace{7cm}
			\textbf {Отчёт по лабораторной работе}\\
			\textbf {Дисциплина:} Телекоммуникационные технологии\\
			\textbf{Тема:} Помехоустойчивые коды
		\end{center} % Конец размещения
		\vspace{8cm} % 
		
		\vfill
		
		\flushleft{Выполнила студентка группы 33501/3} 
		\hfill\parbox{9 cm}{\hspace*{3cm}\hbox to 0cm{\raisebox{-1em}{\small(подпись)}}\hspace*{-0.8cm}\rule{3cm}{0.8pt} Ивашкевич О.А.}\\[0.6cm]
		
		\flushleft{Преподаватель} \hfill\parbox{9 cm}{\hspace*{3cm}\hbox to 0cm{\raisebox{-1em}{\small(подпись)}}\hspace*{-0.8cm}\rule{3cm}{0.8pt} Богач Н.В. }\\[0.6cm]
		
	%	\hfill\parbox{9 cm}{\hspace*{5cm} \today }\\[0.6cm]
		
		
		
		\vspace{\fill}
		\begin{center}
			Санкт-Петербург \\ 2017
		\end{center}
	\end{titlepage}
\newpage
\newpage

\section{Цель работы}
Изучение методов помехоустойчивого кодирования и сравнение их свойств
\section{Постановка задачи}
Провести кодирование/декодирование сигнала кодом Хэмминга 2мя способами с помощью встроенных функций encode/decode, а также через создание проверочной и генераторной матриц и вычисление синдрома. Провести кодирование/декодирование с помощью циклических кодов
Выполнить кодирование/декодирование:	циклическим кодом,кодом БЧХ,кодом Рида-Соломона.

\section{Теоретические сведения}
Кодирование информации необходимо для проверки полученного сообщение на наличие ошибок, так же некторые коды могу исправлять ошибки. Данная функция достигается добавлением в сообщение дополнительной информации (информационной избыточности).
Значительную долю кодов составляют блочные коды. При их применении передаваемое сообщение разбивается на блоки одинаковой длины, после чего каждому блоку сопоставляется код в соответствии с выбранным способом кодирования.
Другая характеристика, позволяющая выделить коды в отдельный класс - цикличность. Цикличность — свойство,суть которого в том что каждая циклическая перестановка кодового слова также является кодовым словом. 
\subsection{Код Хэмминга}
Помимо информационных бит в сообщении передается набор контрольных бит, которые вычисляются как сумма по модулю 2 всех информационных бит, кроме одного. Для $m$ контрольных бит максимальное число информационных бит составляет $2^m-n-1$. Код Хэмминга позволяет обнаружить до двух ошибок при передаче и исправить инверсную передачу одного двоичного разряда. 
\subsection{Циклические коды}
При рассмотрении циклических кодов двичные числа представляют в виде многочлена, степень которого (n-1), где n - длина кодовой комбинации. При таком представлении действия над кодовыми комбинациями сводятся к действиям над многочленами. Эти действия производятся в соотвествии с обычной алгеброй, за исключением того, что приведение подобных членов осуществляются по модулю 2.
\subsection{Коды БЧХ}
Позволяют при необходимости исправлять большее число ошибок в разрядах за счет внесения дополнительной избыточности. Они принадлежат к  категории блочных кодов.
\subsection{Коды Рида-Соломона}
Являются частным случаем кодов БЧХ, работают с недвоичными данными.
\newpage
\section{Ход работы}
\subsection{Код Хэмминга}
Листинг кода приведен ниже:

\begin{lstlisting}[caption=Кодирование Хэмминга]
%By Hamming
msg = [1 0 0 0]  %signal
m=3; n = 2^m-1; k = n-m;
code = encode(msg,n,k,'hamming/ binary ') %coding
decoding = decode(code,n,k,'hamming/binary') %decoding
%By matrix
[parmat,g,n,k] = hammgen(m); % matrix.
trt = syndtable(parmat); % table.
%recd = [1 1 0 1 0 0 0] % Suppose this is the received vector.
recd = msg*g % Suppose this is the received vector.
recd(1)=0; % some error
syndrome = rem(recd * parmat',2);
syndrome_de = bi2de(syndrome,'left-msb');
disp(['Syndrome = ',num2str(syndrome_de),...
      ' (decimal), ',num2str(syndrome),' (binary)'])
corrvect = trt(1+syndrome_de,:) % Correction vector
correctedcode = rem(corrvect+recd,2)
decoding = decode(correctedcode,n,k,'hamming/binary') %decoding
\end{lstlisting}

\begin{lstlisting}[caption=Результат работы encode/decode]
msg = 1     0     0     0
code = 1     1     0     1     0     0     0
decoding = 1     0     0     0
\end{lstlisting}

\begin{lstlisting}[caption= {Результат, полученный с использованием матрицы с исправлением ошибки}] 
recd = 1     1     0     1     0     0     0
Syndrome = 4 (decimal), 1  0  0 (binary)
corrvect = 1     0     0     0     0     0     0
correctedcode = 1     1     0     1     0     0     0
decoding = 1     0     0     0

\end{lstlisting}
\subsection{Циклический код}
Листинг кода приведен ниже:
\begin{lstlisting}[caption=Циклический код]
% Cycle code
message = [1 0 0 0];
clen=6;
mlen=4;
pol = cyclpoly(clen, mlen);
[h_cycle, g_cycle] = cyclgen(clen, pol);
codecycle = message*g_cycle;
codecycle = rem(codecycle, ones(1,clen).*2)
decode_message = [code(4), code(5), code(6), code(7)]
syndrome = rem(codecycle*h_cycle', ones(1,clen-mlen).*2)
\end{lstlisting}

\begin{lstlisting}[caption=Результат циклического кодирования]
codecycle = 1     0     1     0     0     0
decode_message = 1     0     0     0
syndrome = 0     0
\end{lstlisting}

\newpage
\subsection{Код БЧХ}
Листинг кода приведен ниже:
\begin{lstlisting} [caption=Код БЧХ]
msg = [1 0 0 0]
code_msg = comm.BCHEncoder(7,4) 
	decode_msg = comm.BCHDecoder(7,4) 
	temp = msg'; % transpose
	code = step (code_msg , temp(:))' 
	code(2) = ~ code(4) % some error
	decode = step (decode_msg , code')'
\end{lstlisting}
\begin{lstlisting} [caption=Резульат работы кода БЧХ]
code = 1     0     0     0     1     0     1
code = 1     1     0     0     1     0     1
decode =  1     0     0     0
\end{lstlisting}
\subsection{Код Рида-Соломона}
Листинг кода приведен ниже:
\begin{lstlisting} [caption=Код Рида-Соломона]
fprintf('RS code');
code_rs = comm.RSEncoder(6,4);
dec_rs = comm.RSDecoder(6,4);
temp = message';
code = step(code_rs, temp(:))'
decode = step(dec_rs, code')'
\end{lstlisting}
\begin{lstlisting}[caption=Резульат работы кода Рида-Соломона]
code = 1     0     0     0     1     1
decode = 1     0     0     0
\end{lstlisting}

\section{Выводы}
Методы помехоустойчивого кодирования используют избыточную информацию, которая необходима для проверки наличия ошибок в передаваемой информации, что обеспечивает надежность передачи сообщений. Код Хэмминга является самым простым по реализации из рассмотренных способов кодирования. Другие коды имеют более сложную реализацию, но обладают улучшенными корректирующими свойствами и большей надежностью.

\end{document}